\documentclass{article}
\usepackage[utf8]{inputenc}
\usepackage{amsmath}
\usepackage{amsthm}
\usepackage{bm}
\usepackage{amsfonts}
\usepackage{titlesec}
\usepackage[colorlinks, linkcolor=blue]{hyperref}
\usepackage{color}
\usepackage{booktabs}
\usepackage{amsthm}
\usepackage{listings}
\usepackage{exsheets}

%% Font Policy
% vector, \bm{}
% matrix, \bm{}
% space, \mathbb{}

%%%% Define "Lecture" (from redefining "Chapter") %%%% 
% \titleformat{\chapter}[block]{\LARGE\bfseries}{Lecture~\arabic{chapter}.}{1em}{}[]
%%%% ============================================ %%%%

%%% Set up exercise-solution format %%%
%\SetupExSheets{
	%	counter-format = ch.qu ,
	%	counter-within = chapter
	%}
%%% =============================== %%%

\renewcommand{\vec}[1]{\mathbf{#1}}
\newtheorem*{remark}{Remark}
\newtheorem{theorem}{Theorem}[section]
\newtheorem{definition}{Definition}[section]
\newtheorem{problem}{Problem}
\newtheorem{corollary}{Corollary}[section]
\newtheorem{lemma}{Lemma}[section]
\newtheorem{property}{Property}[section]

\author{Fang Zhu}
\title{Lecture 4 The Singular Value Decomposition}

\begin{document}
	\maketitle
	
\section{Prerequisite}
\begin{lemma}
	Given summetrix matrix $\bm{A}$, then the eigenvalues of $\bm{A}$ are real.
\end{lemma}

\begin{theorem}\label{thm:schur-dec-sym-real}
	Given symmetric matrix $\bm{A}$, then $\bm{A}$ can be factored as
	$$
		\bm{A} = \bm{Q} \bm{\Lambda} \bm{Q}^{\star},
	$$
	where
	\begin{itemize}
		\item $\bm{Q}$ is unitary; 
		\item $\Lambda$ is diagonal, with the eigenvalues of $\bm{A}$ on its dagonal.
	\end{itemize}
\end{theorem}
\begin{proof}
	By induction of the dimension of $\bm{A}$.
\end{proof}

\section{Solutions}
\subsection{Exercise 4.1(e)}
First we compute the singular values $\sigma_i$ by finding the eigenvalues of $\bm{A}^\star\bm{A}$:
$$
\bm{A}^\star\bm{A} = \begin{pmatrix}
	2 & 2\\
	2 & 2
\end{pmatrix},
$$
the characteristic polynomial of $\bm{A}^\star \bm{A}$ is
$$
\mathrm{det} (\bm{A}^\star\bm{A} - \lambda \bm{I}) = \lambda(\lambda-4) = 0,
$$
so the singular values are $\sigma_1 = 0, \sigma_2 = 2$.
For $\lambda = 4$, we have
$$
\bm{A}^\star \bm{A} - 4\bm{I}= \begin{pmatrix}
	-2 & 2\\
	2 & -2
\end{pmatrix},
$$
a unit vector in the kernel of the matrix is $\bm{v}_2 = \left(1/\sqrt{2}, 1/\sqrt{2}\right)^T$. For $\lambda = 0$, we have
$$
\bm{A}^\star \bm{A} - 0\bm{I}= \begin{pmatrix}
	2 & 2\\
	2 & 2
\end{pmatrix},
$$
a unit vector in the kernel of the matrix is $\bm{v}_2 = \left(-1/\sqrt{2}, 1/\sqrt{2}\right)^T$. So at this point we know that
$$
\bm{A} = \bm{U\Sigma} \bm{V}^{\star} = \left(\bm{u}_1, \bm{u}_2\right)\begin{pmatrix}
	2 & 0\\
	0 & 0
\end{pmatrix}\begin{pmatrix}
\frac{1}{\sqrt{2}} & \frac{1}{\sqrt{2}}\\
-\frac{1}{\sqrt{2}} & \frac{1}{\sqrt{2}}
\end{pmatrix}.
$$
Finally we can compute $\bm{u}_1$ by the formula $\sigma_i \bm{u}_i = \bm{A} \bm{v}_i$, this gives $\bm{u}_i = \left(\sqrt{2}/2, \sqrt{2}/2\right)$, then by $\bm{u}_2^\star \bm{u}_1 = 0$ and$\| \bm{u}_2\|_2 = 1$ we can get a $\bm{u}_2 = \left(-\sqrt{2}/2, \sqrt{2}/2\right)$. So in this full glory the SVD is
$$
\bm{A} = \bm{U\Sigma} \bm{V}^{\star} = \begin{pmatrix}
	\frac{\sqrt{2}}{2} & -\frac{\sqrt{2}}{2}\\
	\frac{\sqrt{2}}{2} & \frac{\sqrt{2}}{2}
\end{pmatrix}\begin{pmatrix}
	2 & 0\\
	0 & 0
\end{pmatrix}\begin{pmatrix}
	\frac{1}{\sqrt{2}} & \frac{1}{\sqrt{2}}\\
	\frac{-1}{\sqrt{2}} & \frac{1}{\sqrt{2}}
\end{pmatrix} = \bm{U\Sigma V}^{\star}.
$$

\subsection{Exercise 4.2}
Assume that
$$
\bm{A} = \begin{pmatrix}
	a_{11} & a_{12} & \cdots a_{1n} \\
	a_{21} & a_{22} & \cdots a_{2n} \\
	\vdots & \vdots & \ddots \vdots \\
	a_{m1} & a_{m2} & \cdots a_{mn}
\end{pmatrix} = \begin{pmatrix}
	\bm{\alpha}_1^T \\
	\bm{\alpha}_1^T\\ 
	\vdots \\
	\bm{\alpha}_m^T\end{pmatrix}
$$
then we can get that matrix $\bm{B}$
$$
\bm{B} = \begin{pmatrix}
	a_{m1} & \cdots & a_{21} & a_{1n} \\
	a_{m2} & \cdots & a_{22} & a_{12}\\
	\vdots & \vdots & \ddots \vdots \\
	a_{mn} & \cdots & a_{2n} & a_{1n}
\end{pmatrix} = \left(\bm{\alpha}_m, \cdots, \bm{\alpha}_2, \bm{\alpha}_1 \right)
$$
that is 
$$
\bm{B} = \bm{A}^T \begin{pmatrix}
	0 & 0 & \cdots 1 \\
	\vdots & \vdots & \ddots \vdots \\
	0 & 1 & \cdots 0 \\
	1 & 0 & \cdots 0
\end{pmatrix} = \bm{A}^T \bm{P}
$$
it is clear that $\bm{P}$ is a othogonomal matrix since $\bm{P}\bm{P}^T = \bm{P}^T \bm{P} = \bm{I}_m$, then
$$
\bm{B} \bm{B}^T = \bm{A}^T \bm{P} \bm{P}^T \bm{A} = \bm{A}^T \bm{A},
$$
which means that $\bm{B}$ and $\bm{A}$ have that same sigular values.

\subsection{Exercise 4.3}
See \href{./exercise_4_3.m}{matlab code}.

\subsection{Exercise 4.4}
If $\bm{A}, \bm{B}$ are unitary quivalent, we can get that $\bm{A}, \bm{B}$ have the same singular values
by the same argument in 4.2. It is clear that $\bm{I}, -\bm{I}$ have the same singular values, but $\bm{I}$ and $-\bm{I}$ can't be unitary equivalent since $\bm{I} \neq \bm{Q} \left( - \bm
I\right) \bm{Q}^\star = -\bm{I}$.

\subsection{Exercise 4.5}
If $\bm{A}$ is a real matrix, then by \autoref{thm:schur-dec-sym-real}, $\bm{A^\star A}$ is a real symmetric matrix and $\bm{A^\star A}$ has a real eigen decomposition
$$
\bm{A}^\star \bm{A} = \bm{V} \bm{\Lambda} \bm{V}^{\star}
$$
where $\Lambda$ is a diagonal matrix with its entries are eigenvalues of $\bm{A^\star A}$, then we can get $\bm{U}$ by $\bm{A} \bm{v}_i = \sigma_i \bm{u}_i$, we get the real SVD of $\bm{A}$:
$$
\bm{A} = \bm{U} \bm{\Sigma} \bm{V}^\star.
$$
\end{document}