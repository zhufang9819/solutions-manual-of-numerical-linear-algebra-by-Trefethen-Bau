\documentclass{article}
\usepackage[utf8]{inputenc}
\usepackage{amsmath}
\usepackage{amsthm}
\usepackage{bm}
\usepackage{amsfonts}
\usepackage{titlesec}
\usepackage{hyperref}
\usepackage{color}
\usepackage{booktabs}
\usepackage{amsthm}
\usepackage{listings}
\usepackage{exsheets}

%% Font Policy
% vector, \bm{}
% matrix, \bm{}
% space, \mathbb{}

%%%% Define "Lecture" (from redefining "Chapter") %%%% 
% \titleformat{\chapter}[block]{\LARGE\bfseries}{Lecture~\arabic{chapter}.}{1em}{}[]
%%%% ============================================ %%%%

%%% Set up exercise-solution format %%%
%\SetupExSheets{
%	counter-format = ch.qu ,
%	counter-within = chapter
%}
%%% =============================== %%%

\renewcommand{\vec}[1]{\mathbf{#1}}
\newtheorem*{remark}{Remark}
\newtheorem{theorem}{Theorem}[section]
\newtheorem{definition}{Definition}[section]
\newtheorem{problem}{Problem}
\newtheorem{corollary}{Corollary}[section]
\newtheorem{lemma}{Lemma}[section]
\newtheorem{property}{Property}[section]

\author{Fang Zhu}
\title{Lecture 1 Matrix-Vector Multiplication}

\begin{document}
	\maketitle

\section{Prerequisite}
todo...

\section{Solutions}
\subsection{Exercise 1.3}
    \begin{proof}
    We denote a non-singular matrix $\bm{R}$ as
    $$
    \bm{R} = \left(\begin{array}{ccc}
         r_{11} & \cdots & r_{1m}  \\
         \vdots & \ddots & \vdots \\
         0 & \cdots & r_{mm}
    \end{array}\right),
    $$
    it is clear that $r_{ii} \neq 0$, otherwise $\bm{R}$ is singular. Since $\bm{R}$ is non-singular, we assume that
    \[ \bm{I} = (\bm{e}_1, \bm{e}_2, \cdots, \bm{e}_{m}) = (\bm{a}_1, \bm{a}_2, \cdots, \bm{a}_n)\left(\begin{array}{ccc}
         r_{11} & \cdots & r_{1m}  \\
         \vdots & \ddots & \vdots \\
         0 & \cdots & r_{mm}
    \end{array}\right)
    \]
    where $(\bm{a}_1, \cdots, \bm{a}_n) = \bm{R}^{-1}$. To show $\bm{R}^{-1}$ is upper-triangular, we work by induction. To begin with, we have $\bm{e}_1 = r_{11} \bm{a}_1$ and hence $\bm{a_1} = r_{11}^{-1} \vec{e}_1$ has \textit{zero entries} except the first one. For convenience, we denote by $\mathbb{C}^{m}_{k}$ the column space
    \[ \mathbb{C}_{k}^{m} = \{ \bm{v} = (v_1, \cdots, v_{k}, 0,\cdots, 0)^{T}, v_{i} \neq 0 ~(1\leq i \leq k)\}, \]
    Then
    \[ \mathbb{C}^{m}_1 \subset \mathbb{C}^{m}_{2} \cdots \mathbb{C}^{m}_{m}  = \mathbb{C}^{m}. \]
    We have shown that $a_1 \in \mathbb{C}^{m}(1)$, assume that for any $k \leq s$, we have that $\mathbf{a}_k \in \mathbb{C}^{m}_{k}$. Then by equation \textit{Page 8, (1.8)}, we have
    \[\bm{e}_{s+1} = \sum_{k=1}^{m} \bm{a}_k r_{k,s+1}.\]
    Note that $r_{k,s+1} = 0, ~\forall k > s+1$, then
    \[ \sum_{k=1}^{m} \bm{a}_k r_{k, s+1} = \sum_{k = 1}^{s} \bm{a}_k r_{k, s+1} + \bm{a}_{s+1} r_{s+1, s+1} = \bm{e}_{s+1},\]
    Therefore
    \[ \bm{a}_{s+1} = r_{s+1,s+1}^{-1} (\vec{e}_{s+1} - \sum_{k=1}^{s} \bm{a}_{k} r_{k,s+1}) \in \mathbb{C}^{m}_{s+1}\]
    By induction, we have proved that $\bm{a}_{k} \in \mathbb{C}^{m}_{k}$ for $1 \leq k \leq m$, which is equivalent to the fact that $\bm{R}^{-1}$ is upper-triangular.
    \end{proof}
    \subsection{Exercise 1.4(a)}
    \begin{proof}
    Denote the column vectors $(c_1, \cdots, c_n)^{T}$, $(d_1, \cdots, d_n)^{T}$ by notations $\bm{c}$ and $\bm{d}$, let $\bm{F}$ be the matrix whose $(i,j)$ entry is $f_j(i)$. Then, the given condition can be rephrased as: ForAll $\bm{d} \in \mathbb{C}^8$, there must exist a vector $\bm{c}$ such that $\bm{F} \bm{c} = \bm{d}$. This means that 
    $$
    \mathrm{range} \{\bm{F} \} = \mathbb{C}^8, 
    $$ 
    which implies that $\bm{F}$ has full rank by \textit{theorem 1.3}. Furthermore, $\mathbf{F}$ is non-singular. Therefore
    \[ \bm{c} = \bm{F}^{-1} \bm{d}\]
    and hence $\bm{d}$ determines $\bm{c}$ uniquely.\\
    \end{proof}

\subsection{Exerciese 1.4(b)}
    The given condition can be reformatted as
    $$
    \bm{A} \bm{d} = \bm{c}.
    $$
    Note that $\bm{c} = \bm{F}^{-1} \bm{d}$, then 
    $$
    \bm{A} \bm{d}  = \bm{c} = \bm{F}^{-1} \bm{d},
    $$
    then we have
    $$
    (\bm{FA} - \bm{I})\bm{d } = \bm{0}, 
    $$
    note that this equation above is true for any $d \in \mathbb{C}^{8}$, then $\bm{FA} - \bm{I}$ must be \textit{zero matrix}, which is $\bm{FA} = \bm{I}$. Hence the $i,j$ entry of $\bm{A}^{-1}$ is the $i,j$ entry of $\bm{F}$ we defined in 1.4(a).

\end{document}